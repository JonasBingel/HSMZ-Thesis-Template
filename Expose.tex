%-----------------------------------------------------------------------------------------
%  Dieses Projekt wurde erstellt mit der LaTeX Vorlage von Jonas Bingel
%  Die LaTeX Vorlage kann heruntergeladen werden unter: github.com/JonasBingel/HSMZ-Thesis-Template
%  
% -----------------------------------------------------------------------------------------

% Erklärender Text zu dieser Datei --------------------------------------------------------
% Die Datei Arbeit.tex ist die Hauptdatei, die dem LaTeX-Kompiler übergeben wird.
% Diese Datei ist zu verwenden, wenn das Expose erstellt werden soll.
% In dieser Datei wird 
%  - das Dokument erstellt und alle anderen tex-Dateien werden geladen.
%  - die .bib-Datei eingebunden zur Bereitstellung der Quellen
%  - die Struktur des Dokuments wird definiert
% -----------------------------------------------------------------------------------------

\documentclass[
ngerman, % neue Deutsche Rechtschreibung
headings=normal, % normale Headings
captions=tableheading, % Positioniert Captions über Tabellen
listof=totoc,
bibliography=totoc,
%usegeometry, % Auskommentieren wenn Package geometery verwendet wird
overfullrule, % entfernen nach abschluss der bearbeitung
% Default Values, die scrreprt nutzt
%oneside, % Einseitige Seitengenerierung
%11pt, % Font size
% a4paper, % paper
]{scrreprt}
\KOMAoptions{%
  DIV=12,
  parskip=half*,
% %BCOR=korrektur, % absoluter Wert der Bindekorrektur
}

% Erklärender Text zu dieser Datei --------------------------------------------------------
% Die Datei Metadaten.tex dient der Definition von Metadaten zur vorliegenden Arbeit.
% Dies umfasst Informationen zu
%  - der Arbeit selbst(Titel, Art)
%  - dem Betreuer
%  - dem Hochschule
%  - ggf. kooperierenden Unternehmen
% Die Metadaten werden genutzt in Titelseite.tex, Sperrvermerk.tex und Erklaerung.tex
% Ferner werden die Metadaten des exportierten PDFs entsprechend gesetzt - eine Konfiguration ist in Packages.tex möglich.
% -----------------------------------------------------------------------------------------

% Meta-Daten zu dieser Arbeit ------------------------------------------------------------
\newcommand{\art}{Exposé zur Bachelorarbeit\xspace}
\newcommand{\titel}{Titel der Arbeit\xspace}
\newcommand{\hochschule}{Hochschule Mainz}
\newcommand{\hochschulezusatz}{University of Applied Sciences}
\newcommand{\logo}{logo_hsmz.png}
\newcommand{\fachbereich}{Wirtschaft}
\newcommand{\studiengang}{Wirtschaftsinformatik B.Sc. dual}

\newcommand{\autor}{Vorname Nachname}
\newcommand{\strasseAutor}{Straße}
\newcommand{\stadtAutor}{PLZ Ort}
\newcommand{\matrikelnr}{123456}

\newcommand{\unternehmen}{Unternehmen}
\newcommand{\datumAblaufSperrvermerk}{01.01.1970}

\newcommand{\betreuer}{Prof. Dr. Betreuer}
\newcommand{\datumAbgabe}{01.01.1970}
\newcommand{\ort}{Mainz}


% Erklärender Text zu dieser Datei --------------------------------------------------------
% Die Datei Packages.tex dient als zentrale Datei, in der alle genutzten Packages geladen und konfiguriert werden.
% Basierend auf den Anforderungen sind ggf. Konfigurationen vereinzelt zu ändern - entsprechende Stellen sind mit %TODO gekennzeichnet.
% Beispiele für solche Anforderungen sind
%  - Package biblatex: Definition, des Zitationsstils und Aufbau des Literaturverzeichnisses; Default ist IEEE, Konfiguration für APA ist hinterlegt
%  - Package minted: Definition, wie Source Code von Sprachen dargestellt werden soll
%  - Zur korrekten Darstellung des Inhaltsverzeichnisses die "breiteste" Seitenzahl angeben
%
% HINWEIS: Die Reihenfolge, in der Packages geladen werden ist wichtig. Hinweise aus den Dokumentationen der einzelnen Packages sind daher unbedingt zur berücksichtigen, wenn weitere Packages aufgenommen werden sollen.
% -----------------------------------------------------------------------------------------
\usepackage[utf8]{inputenc}
\usepackage[T1]{fontenc}
\usepackage{babel}
\usepackage{lmodern}
\usepackage{xspace} % Leerzeichen hinter parameterlosen Makros nicht als Endzeichen interpretieren
\usepackage{graphicx} %Abbildungen
\usepackage{pdfpages} %Einfügen von PDF-Seiten
\includepdfset{scale=0.7, frame, pagecommand={\thispagestyle{plain}}}
\graphicspath{{04_Artefakte/01_Abbildungen/}}
\usepackage{caption} % Bildunterschriften
\usepackage{subcaption} % https://www.ctan.org/pkg/subcaption

\usepackage{tabularx} % Tabellen
\usepackage{booktabs} % Bessere Tabellen
\usepackage[longtable]{multirow}
\usepackage{longtable}

\usepackage{amsmath}
\usepackage{amsfonts}
\usepackage{bbm}

\usepackage{xcolor}
%\usepackage{chngcntr} % fortlaufende Nummerierung von Fußnoten

\usepackage[colorinlistoftodos]{todonotes} % to disable todos use option disable, alternatively use obeyDraft or obeyFinal
\usepackage{blindtext}

\usepackage{microtype} % auskommentieren moeglich, wenn Typografie nicht zufriedenstellend

\usepackage[bottom]{footmisc} % https://golatex.de/viewtopic.php?f=21&t=24052
% Verlinkung und PDF Bookmarks https://tex.stackexchange.com/a/83051
\usepackage[nospace]{varioref}
% extensions keep all links black, only urls are blue: https://tex.stackexchange.com/a/401885/220502
\usepackage[hidelinks, colorlinks, allcolors=., urlcolor=blue]{hyperref}
\usepackage{bookmark}
\hypersetup{
    pdftitle={\titel},
    pdfauthor={\autor},
    pdfcreator={\autor},
    pdfsubject={\titel},
    pdfkeywords={\titel},
}
\usepackage{cleveref}

% Literaturverzeichnis und Quellenverwaltung mittels biblatex
%TODO Änderungen des Zitationsstils und Literaturverzeichnisses
% Zitationsstil hier auswählen:

\newcommand{\ZitatStil}{apa} % (apa oder ieee)

\ifthenelse{\equal{\ZitatStil}{apa}}{%
  \usepackage[sortlocale=auto,sorting=nyt,style=apa]{biblatex}%
}{
\ifthenelse{\equal{\ZitatStil}{iee}}{%
  \usepackage[backend=biber,style=ieee,dashed=false,]{biblatex}%
}}


% Verwalten von Abkuerzungen und einem Abkuerzungsverzeichnis
\usepackage[printonlyused]{acronym} % https://www.ctan.org/pkg/acronym

% Packages for forcing floats https://robjhyndman.com/hyndsight/latex-floats/
\usepackage{afterpage}
\usepackage[section]{placeins}
\usepackage{censor}


% Erstellen eines Symbolverzeichnisses
\usepackage[intoc, german, stdsubgroups]{nomencl} % https://www.ctan.org/pkg/nomencl
\usepackage{siunitx}
\newcommand{\nomunit}[1]{%
    \renewcommand{\nomentryend}{\hspace*{\fill}\si{#1}}}
\makenomenclature

%TODO Definition der Darstellung von Source Code
\usepackage[newfloat, cache]{minted} % https://www.ctan.org/pkg/minted
\SetupFloatingEnvironment{listing}{name=Listing, placement=b}
\SetupFloatingEnvironment{listing}{listname={Listingverzeichnis}}
\setminted[java]{linenos, fontsize=\footnotesize, frame=lines, breaklines, breakbefore={.}}
\usemintedstyle{borland}
% new environment for listings longer than one page; https://tex.stackexchange.com/a/53540
\newenvironment{longlisting}{\captionsetup{type=listing}}{}
\usepackage{csquotes}


% Darstellung von Algorithmen und Pseudocode
% do NOT use option naturalnames if you compile with pdflatex and use hyperref
\usepackage[linesnumbered, commentsnumbered, ruled]{algorithm2e} 
\renewcommand{\listalgorithmcfname}{Algorithmusverzeichnis}
\addtotoclist[float]{loa}
\renewcommand\listofalgorithms{\listoftoc[{\listalgorithmcfname}]{loa}}
%\SetAlFnt{\small \normalfont \sffamily} 
\SetAlFnt{\footnotesize} 


% Erstellung neuer Verzeichnisse; Code weitgehend von Markus Kohm und komascript.de 
\DeclareNewTOC[%
  owner=anhang,
  listname={Anhangsverzeichnis},% Titel des Verzeichnisses
]{atoc}

\DeclareNewTOC[%
  type=equation,
  listname={Formelverzeichnis},
  tocentrynumwidth=2.3em,
]{loe}

\makeatletter
\renewcommand\@pnumwidth{2em} % vermeiden von overful hbox im Inhaltsverzeichnis

\AfterTOCHead[atoc]{\let\if@dynlist\if@tocleft}
\newcommand*{\useappendixtocs}{%
  \renewcommand*{\ext@toc}{atoc}%
  \scr@ifundefinedorrelax{hypersetup}{}{% damit es auch ohne hyperref funktioniert
    \hypersetup{bookmarkstype=atoc}%
  }%
}
\newcommand*{\usestandardtocs}{%
  \renewcommand*{\ext@toc}{toc}%
  \scr@ifundefinedorrelax{hypersetup}{}{% damit es auch ohne hyperref funktioniert
    \hypersetup{bookmarkstype=toc}%
  }%
  \renewcommand*{\ext@figure}{lof}%
  \renewcommand*{\ext@table}{lot}%
}
\scr@ifundefinedorrelax{ext@toc}{%
  \newcommand*{\ext@toc}{toc}
  \renewcommand{\addtocentrydefault}[3]{%
    \expandafter\tocbasic@addxcontentsline\expandafter{\ext@toc}{#1}{#2}{#3}%
  }
}{}
\newcommand*{\@currententry}{}
% Zwei amsmath-Anweisungen ändern:
\g@addto@macro\make@display@tag{\set@currententry}%
\def\tagform@#1{\maketag@@@{(\ignorespaces#1\unskip\@@italiccorr)}%
  \set@currententry}
\newcommand*{\set@currententry}{%
  \typeout{set current entry}%
  \ifx\@currententry\@empty\else
    \addcontentsline{loe}{equation}{\protect\numberline{\@currentlabel}%
      \@currententry}%
    \global\let\@currententry\@empty
  \fi
}
% Neue Benutzeranweisung
\newcommand*{\equationentry}[1]{%
  \gdef\@currententry{#1}%
}

\makeatother

\usepackage{xpatch}
\xapptocmd\appendix{%
  \useappendixtocs
  \pdfbookmark{Anhangsverzeichnis}{anhangsverzeichnis}
  \listofatocs
  \addcontentsline{toc}{chapter}{Anhangsverzeichnis}
  \bookmarksetupnext{level=-1}
}{}{}

% Pakete, die fuer Informatik Sinn ergeben koennten
% \usepackage{bytefield} % illustration of fields of data https://www.ctan.org/pkg/bytefield


% Uebersetzung fuer Eintraege im Abkuerzungsverzeichnis - Code uebernommen von https://tex.stackexchange.com/a/135507
\makeatletter
\newcommand{\acroforeign}[1]{}

% patch the environment to print the foreign definition:
\AtBeginEnvironment{acronym}{%
  \def\acroforeign#1{ (#1)}%
}

% patch the acronym definition to safe the foreign definition:
\expandafter\patchcmd\csname AC@\AC@prefix{}@acro\endcsname
  {\begingroup}
  {\begingroup\def\acroforeign##1{\csdef{ac@#1@foreign}{##1, }}}
  {}
  {\fail}

% %   renew the first output to include the foreign definition if given:
\renewcommand*{\@acf}[2][\AC@linebreakpenalty]{%
  \ifAC@footnote
    \acsfont{\csname ac@#2@foreign\endcsname\AC@acs{#2}}%
    \footnote{\AC@placelabel{#2}\AC@acl{#2}{}}%
  \else
    \acffont{%
      \AC@placelabel{#2}\AC@acl{#2}%
      \nolinebreak[#1] %
      \acfsfont{(\acsfont{\csname ac@#2@foreign\endcsname\AC@acs{#2}})}%
    }%
  \fi
  \ifAC@starred\else\AC@logged{#2}\fi
}
\makeatother

% Adjusting the width that is reserved for the  pagenumber in listings
% https://projekte.dante.de/DanteFAQ/Verzeichnisse#2
% https://de.comp.text.tex.narkive.com/fAP3Znev/overfull-hbox-im-inhaltsverzeichnis
\makeatletter
\AtBeginDocument{
\newlength{\mylen}
\setlength{\mylen}{\widthof{XVIII}} %TODO hier Text eintragen, der der breitesten Seitennummer im Inhaltsverzeichnis oder einem der anderne Verzeichnisse entspricht
\renewcommand*\@pnumwidth{\the\mylen}
}
\makeatother

% Erklärender Text zu dieser Datei --------------------------------------------------------
% Die Datei Befehle.tex dient der Definition von Custom-Anweisungen, sodass diese in den restlichen Dateien genutzt werden können. 
% Custom-Anweisungen eigenen sich inbesondere als Variablen für Fachbegriffe, sodass sichergestellt ist, dass diese in der gesamten Arbeit einheitlich geschrieben sind und zentral geändert werden können.
% Zudem enthält die Datei bereits Anweisungen für häufig genutzte Abkürzungen.
% -----------------------------------------------------------------------------------------

% Befehle zu Abkuerzungen --------------------------------------------------------------
% Die gelisteten Befehle für Abkürzungen stammen aus github.com/StefanMacke/latex-vorlage-fiae/blob/master/Allgemein/Befehle.tex

% Die Anweisung \, erzeugt einen kurzen Abstand und wird bei Abkuerzungen oder zwischen Zahlen und Masseinheiten verwendet
\newcommand{\bs}{$\backslash$\xspace}
\newcommand{\ua}{\mbox{u.\,a.}\xspace}
\newcommand{\oa}{\mbox{o.\,a.}\xspace}
\newcommand{\bspw}{bspw.\xspace}
\newcommand{\bzw}{bzw.\xspace}
\newcommand{\ca}{ca.\xspace}
\newcommand{\dahe}{\mbox{d.\,h.}\xspace}
\newcommand{\etc}{etc.\xspace}
\newcommand{\eur}[1]{\mbox{#1\,\texteuro}\xspace}
\newcommand{\evtl}{evtl.\xspace}
\newcommand{\ggfs}{ggfs.\xspace}
\newcommand{\Ggfs}{Ggfs.\xspace}
\newcommand{\gqq}[1]{\glqq{}#1\grqq{}} % Anführungszeichen oben und unten
\newcommand{\idR}{i.d.R.\xspace}
\newcommand{\inkl}{inkl.\xspace}
\newcommand{\insb}{insb.\xspace}
\newcommand{\usw}{usw.\xspace}
\newcommand{\Vgl}{Vgl.\xspace}
\newcommand{\sogn}{sogn.\xspace}
\newcommand{\zB}{\mbox{z.\,B.}\xspace}
\newcommand{\engl}{engl.\xspace}
\newcommand{\dt}{dt.\xspace}

% Befehle zu Fachbegriffen --------------------------------------------------------------
% Es ist empfehlenswert alle Fachbegriffe als Anweisung zu deklarieren, damit diese einheitlich sind und zentral geändert werden können

% In der folgenden Zeile ein Beispiel für den Begriff "Q-Learning", sodass im Text die Anweisung \qlearning genutzt werden kann, welche zu "Q-Learning" übersetzt wird 
%\newcommand{\qlearning}{Q-Learning\xspace}


% Custom Anweisungen ------------------------------------------------------------------
% Anweisung um mehrere PDF-Seiten einzufügen
\newcommand\includepdfWithChapter[4]{%
\includepdf[pages={#1}, pagecommand={\chapter{#3}}]{#4}
\includepdf[pages={#2}]{#4}
}



% Typesetting options

\widowpenalty=10000     % Hurenkinder
\clubpenalty=10000      % Schusterjungen

\addbibresource{sample.bib}
% Wegen dem folgenden Befehl werden im Literaturverzeichnis alle Quellen gelistet, die in der .bib Datei enthalten sind. 
% Wird der Befehl entfernt, werden nur die Quellen gelistet, die auch zitiert werden.
% Der Befehl ist also beim Expose dringend notwendig.
\nocite{*} 
\newcounter{romanConsecutive}

\begin{document}
% Erklärender Text zu dieser Datei --------------------------------------------------------
% Die Datei ToDo.tex dient der Generierung der ToDo-Liste und Auflistung allgemeiner ToDos.
% Zum Erstellen der ToDos wird das Package "todonotes" verwendet.
% Wenn todonotes nicht angezeigt oder genutzt werden sollen, muss in Packages.tex die Option "disable" beim Package "todonotes" eingetragen werden.
% -----------------------------------------------------------------------------------------
\listoftodos
% Die folgenden Todos enthalten Hinweise, der LaTeX-Vorlage für Aktionen, die bei Erstellung einer "Abgabe PDF" notwendig sind.
\todo[inline, color=green]{Bei Abgabe: Anweisung nocite  in Bachelorarbeit.tex entfernen}
\todo[inline, color=green]{Bei Abgabe: In Bachelorarbeit.tex und Expose.tex Dokumentenoption overfullrule entfernen und die Option final eintragen}
\todo[inline, color=green]{Bei Abgabe: In Packages.tex beim Package todonotes die Option disable eintragen, um Todos zu deaktivieren}
\todo[inline, color=green]{Bei Abgabe: In Packages.tex beim TODO die größte Seitennummer eintragen}
\todo[inline, color=green]{Wenn eine Version der Arbeit erstellt wird, die gedruckt werden soll in Packages.tex beim Package hyperref die Option urlcolor=blue entfernen}



\pagestyle{empty}
\renewcommand*{\chapterpagestyle}{empty}
% Einkommentieren, wenn Sperrvermerk notwendig ist 
%% Erklärender Text zu dieser Datei --------------------------------------------------------
% Die Datei 01_Sperrvermerk.tex dient der Definition eines optionalen Sperrvermerks, der vor der Arbeit 
% Der derzeit dargestellte Sperrvermerk entspricht der Vorgabe der HS Mainz und wird mit Angaben aus Metadaten.tex gefüllt
% -----------------------------------------------------------------------------------------
\pdfbookmark{Sperrvermerk}{sperrvermerk}
\addchap*{Sperrvermerk}
Die vorliegende \art mit dem Titel \titel enth"alt interne und vertrauliche Daten des Unternehmens/der Einrichtung \unternehmen.

Die \art darf nur den Gutachtern (insbesondere Erst- und Zweitgutachtern), den Mitgliedern der Prüfungsorgane (einschließlich Beisitzer \& Plagiatskontrolle) sowie den in einem eventuellen Rechtschutzverfahren Betrauten zug"anglich gemacht werden.

Im "Ubrigen ist eine Ver"offentlichtung und Vervielf"altigung der Abschlussarbeit -- auch in Auszügen -- nicht gestattet. Vorbehaltlich der Vorschriften zum Pr"ufungsverfahren und der Pr"ufung bedarf eine Einsichtnahme in die Arbeit durch Dritte einer ausdr"ucklichen Genehmigung der Verfasserin/des Verfassers sowie des \oa Unternehmens.

%Diese Geheimhaltungsverpflichtung gilt bis zum Ablauf des \datumAblaufSperrvermerk.

% Erklärender Text zu dieser Datei --------------------------------------------------------
% Die Datei 02_Titelseite.tex dient der Definition der Titelseite.
% Alle Angaben auf der Titelseite werden mit Informationen aus Metadaten.tex gefüllt
% Solange das Design der Titelseite nicht angepasst werden soll, sind in dieser Datei keine manuellen Änderungen notwendig.
% -----------------------------------------------------------------------------------------
\begin{titlepage}

\begin{minipage}{\textwidth}
		\noindent \hfill \includegraphics{\logo}
\end{minipage}
\vspace{6em}

\begin{center}
    {\huge \art}
    
    {\Large Studiengang \studiengang}
    
    \vspace{4em}
    
    \textbf{{\Large \titel}}
    
    \vspace{4em}
    
    \hochschule
    
    \hochschulezusatz

    Fachbereich \fachbereich
    
    \vspace{6em}

	\begin{minipage}{\textwidth}
		\begin{tabbing}
		
		Vorgelegt von:  \hspace*{2em}\= Philipp Gremeyer (949899) \\%\autor \\
        \> Patrick Pfurtscheller (949902) \\
        \> Subhan Ullah () \\
		% \> \strasseAutor \\
        % \> \stadtAutor \\
        % \> Matrikel-Nr. \matrikelnr \\
        Vorgelegt bei: \> \betreuer \\
        Eingereicht am: \> \datumAbgabe
		\end{tabbing}

	\end{minipage}
\end{center}
\end{titlepage}

% Einkommentieren, wenn Erklärung notwendig sein sollte
%% Erklärender Text zu dieser Datei --------------------------------------------------------
% Die Datei 03_Erklaerung.tex dient der Definition der Eidesstattlichen Erklärung.
% Die Datei enthält den Text, der im Leitfaden der Hochschule Mainz vorgegeben ist.
% Wenn ein Bild einer Unterschrift eingefügt werden soll, ist an der Stelle %TODO dem Hinweis zu folgen.
% -----------------------------------------------------------------------------------------
\pdfbookmark{Erkl"arung}{erklaerung}
\addchap*{Erkl"arung}
Hiermit erkl"are ich, dass ich die vorliegende \art
\begin{quote}
\textbf{\titel}   
\end{quote}

selbstst"andig und ohne fremde Hilfe angefertigt habe. 
Ich habe dabei nur die in der Arbeit angegebenen Quellen und Hilfsmittel benutzt.

Zudem versichere ich, dass ich weder diese noch inhaltlich verwandte Arbeiten als Pr"ufungsleistung in anderen F"achern eingereicht habe oder einreichen werde.

\vspace{3em}
%TODO Wenn eine Unterschrift eingefügt werden soll, muss nachfolgender Kommentar eingefügt werden
% Einfuegen der Unterschrift als Datei "signature.png", Scale muss ggf. angepasst werden
% \begin{figure}[h]
%     \hspace{9cm}
%     \includegraphics[scale=0.2]{04_Artefakte/01_Abbildungen/signature.png}
% \end{figure}
\ort, den \datumAbgabe \hfill \autor 



\pagestyle{plain}
\renewcommand*{\chapterpagestyle}{plain}
\pagenumbering{Roman}

\setcounter{romanConsecutive}{\value{page}}
\pagenumbering{arabic}
% Erklärender Text zu dieser Datei --------------------------------------------------------
% Die Datei 00_Inhalt_Expose.tex dient dem Schreiben des Expose.
% Die Vorlage enthält bereits typische Abschnitte eines Exposes, sodass diese nur noch ausgefüllt werden müssen.
%
% Weitere Abschnitte sind möglich und sollten mit dem Betreuer abgestimmt werden.
% Beispiele für weitere Abschnitte sind: Relevanz und Aktualität des Themas, Theoretischer Hintergrund oder Zeitplanung
% -----------------------------------------------------------------------------------------

\section*{Beschreibung des Themas}
In diesem Abschnitt wird das Thema in mindestens 2-3 Sätzen beschrieben. 
Ziel ist es ein Verständnis für das Thema zu geben und dieses grob zu verorten.

\section*{Forschungsfrage}
In diesem Abschnitt wird die Forschungsfrage der Arbeit formuliert. 
Diese sollte von praktischer oder wissenschaftlicher Relevanz sein und einen allgemeinen Charakter haben.
Im Zuge einer Forschungsfrage können auch weitere Fragestellungen betrachtet werden, die gelistet werden sollten. 
\begin{itemize}
    \item Untergeordnete Fragestellung 1
    \item Untergeordnete Fragestellung 2
    \item Untergeordnete Fragestellung 3
\end{itemize}

\section*{Methodik und Vorgehen}
Beschreibung der Vorgehensweise, die zur Beantwortung der Forschungsfrage genutzt wird. 
Beispielsweise Literaturuntersuchung, Implementierung von Algorithmen oder die Auswertung von Fallbeispielen.

\pagebreak
\section*{Vorl"aufige Gliederung}
% Die nachfolgende Gliederung ist nur ein Beispiel, damit die Befehle zur Erstellung der Gliederung gezeigt werden können.
% Die folgenden zwei Zeilen dienen der richtigen Nummerierung der Aufzählungsebenen
\renewcommand{\labelenumii}{\arabic{enumi}.\arabic{enumii}}
\renewcommand{\labelenumiii}{\arabic{enumi}.\arabic{enumii}.\arabic{enumiii}}
\begin{enumerate}
    \item Einleitung
        \begin{enumerate}
        \item Motivation
        \item Ziele und Forschungsfrage
        \item Aufbau der Arbeit
        \end{enumerate}
    \item Grundlagen
        \begin{enumerate}
        \item Reinforcement Learning
            \begin{enumerate}
            \item Begriffserkl"arung und Verortung
            \item Markov Decision Process
            \item Dynamic Programming
            \item Temporal Difference Learning
            \item Exploitation vs Exploration
            \end{enumerate}
        \item Q-Learning
        \item SARSA      
        \item MiniMax-Algorithmus
        \item Tic-Tac-Toe
            \begin{enumerate}
            \item Spielerkl"arung
            \item Anwendbare Reinforcement Learning Algorithmen
            \end{enumerate}
        \end{enumerate}
    \item Methodik und Funktionsweise der Algorithmen
        \begin{enumerate}
        \item Spielfeld
        \item Agent Q-Learning
        \item Agent SARSA
        \item Trainingsaufbau und Evaluationsmetriken
        \end{enumerate}
    \item Implementierung
    \item Diskussion und Auswertung der Ergebnisse
        \begin{enumerate}
        \item Auswertung des Q-Learning Agenten
        \item Auswertung des SARSA Agenten
        \end{enumerate}
    \item Konklusion
        \begin{enumerate}
        \item Beantwortung der Forschungsfragen
        \item Kritische Betrachtung der Inhalte
        \item Anmerkungen f"ur k"unftige Arbeiten
        \end{enumerate}   
\end{enumerate}



\pagenumbering{Roman}

\setcounter{page}{\value{romanConsecutive}}
\printbibliography[title={Literaturverzeichnis}]

\usestandardtocs
\bookmarksetup{startatroot}% siehe bookmark-Anleitung

\end{document}
