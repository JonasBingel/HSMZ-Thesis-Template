% Erklärender Text zu dieser Datei --------------------------------------------------------
% Die Datei 01_Sperrvermerk.tex dient der Definition eines optionalen Sperrvermerks, der vor der Arbeit 
% Der derzeit dargestellte Sperrvermerk entspricht der Vorgabe der HS Mainz und wird mit Angaben aus Metadaten.tex gefüllt
% -----------------------------------------------------------------------------------------
\pdfbookmark{Sperrvermerk}{sperrvermerk}
\addchap*{Sperrvermerk}
Die vorliegende \art mit dem Titel \titel enth"alt interne und vertrauliche Daten des Unternehmens/der Einrichtung \unternehmen.

Die \art darf nur den Gutachtern (insbesondere Erst- und Zweitgutachtern), den Mitgliedern der Prüfungsorgane (einschließlich Beisitzer \& Plagiatskontrolle) sowie den in einem eventuellen Rechtschutzverfahren Betrauten zug"anglich gemacht werden.

Im "Ubrigen ist eine Ver"offentlichtung und Vervielf"altigung der Abschlussarbeit -- auch in Auszügen -- nicht gestattet. Vorbehaltlich der Vorschriften zum Pr"ufungsverfahren und der Pr"ufung bedarf eine Einsichtnahme in die Arbeit durch Dritte einer ausdr"ucklichen Genehmigung der Verfasserin/des Verfassers sowie des \oa Unternehmens.

%Diese Geheimhaltungsverpflichtung gilt bis zum Ablauf des \datumAblaufSperrvermerk.
