\begin{center}
\begin{longtable}{lll}
\caption{Ein Beispiel für eine lange Tabelle} \label{tab:long} \\

\toprule \multicolumn{1}{c}{\textbf{First column}} & \multicolumn{1}{c}{\textbf{Second column}} & \multicolumn{1}{c}{\textbf{Third column}} \\ \midrule 
\endfirsthead

\multicolumn{3}{c}%
{{\bfseries \tablename\ \thetable{} -- fortgesetzt von voriger Seite}} \\
\midrule \multicolumn{1}{c}{\textbf{First column}} & \multicolumn{1}{c}{\textbf{Second column}} & \multicolumn{1}{c}{\textbf{Third column}} \\ \midrule
\endhead

\midrule \multicolumn{3}{r}{{fortgesetzt auf nachfolgender Seite}} \\
\endfoot

\bottomrule
\endlastfoot
One & abcdef ghjijklmn & 123.456778 \\
One & abcdef ghjijklmn & 123.456778 \\
One & abcdef ghjijklmn & 123.456778 \\
One & abcdef ghjijklmn & 123.456778 \\
One & abcdef ghjijklmn & 123.456778 \\
One & abcdef ghjijklmn & 123.456778 \\
One & abcdef ghjijklmn & 123.456778 \\
One & abcdef ghjijklmn & 123.456778 \\
One & abcdef ghjijklmn & 123.456778 \\
One & abcdef ghjijklmn & 123.456778 \\
One & abcdef ghjijklmn & 123.456778 \\
One & abcdef ghjijklmn & 123.456778 \\
One & abcdef ghjijklmn & 123.456778 \\
One & abcdef ghjijklmn & 123.456778 \\
One & abcdef ghjijklmn & 123.456778 \\
One & abcdef ghjijklmn & 123.456778 \\
One & abcdef ghjijklmn & 123.456778 \\
One & abcdef ghjijklmn & 123.456778 \\
One & abcdef ghjijklmn & 123.456778 \\
One & abcdef ghjijklmn & 123.456778 \\
One & abcdef ghjijklmn & 123.456778 \\
One & abcdef ghjijklmn & 123.456778 \\
One & abcdef ghjijklmn & 123.456778 \\
One & abcdef ghjijklmn & 123.456778 \\
One & abcdef ghjijklmn & 123.456778 \\
One & abcdef ghjijklmn & 123.456778 \\
One & abcdef ghjijklmn & 123.456778 \\
One & abcdef ghjijklmn & 123.456778 \\
One & abcdef ghjijklmn & 123.456778 \\
One & abcdef ghjijklmn & 123.456778 \\
One & abcdef ghjijklmn & 123.456778 \\
One & abcdef ghjijklmn & 123.456778 \\
One & abcdef ghjijklmn & 123.456778 \\
One & abcdef ghjijklmn & 123.456778 \\
One & abcdef ghjijklmn & 123.456778 \\
One & abcdef ghjijklmn & 123.456778 \\
One & abcdef ghjijklmn & 123.456778 \\
One & abcdef ghjijklmn & 123.456778 \\
One & abcdef ghjijklmn & 123.456778 \\
One & abcdef ghjijklmn & 123.456778 \\
One & abcdef ghjijklmn & 123.456778 \\
One & abcdef ghjijklmn & 123.456778 \\
One & abcdef ghjijklmn & 123.456778 \\
One & abcdef ghjijklmn & 123.456778 \\
One & abcdef ghjijklmn & 123.456778 \\
One & abcdef ghjijklmn & 123.456778 \\
One & abcdef ghjijklmn & 123.456778 \\
One & abcdef ghjijklmn & 123.456778 \\
One & abcdef ghjijklmn & 123.456778 \\
One & abcdef ghjijklmn & 123.456778 \\
One & abcdef ghjijklmn & 123.456778 \\
One & abcdef ghjijklmn & 123.456778 \\
One & abcdef ghjijklmn & 123.456778 \\
One & abcdef ghjijklmn & 123.456778 \\
One & abcdef ghjijklmn & 123.456778 \\
One & abcdef ghjijklmn & 123.456778 \\
One & abcdef ghjijklmn & 123.456778 \\
One & abcdef ghjijklmn & 123.456778 \\
One & abcdef ghjijklmn & 123.456778 \\
One & abcdef ghjijklmn & 123.456778 \\
One & abcdef ghjijklmn & 123.456778 \\
One & abcdef ghjijklmn & 123.456778 \\
One & abcdef ghjijklmn & 123.456778 \\
One & abcdef ghjijklmn & 123.456778 \\
One & abcdef ghjijklmn & 123.456778 \\
One & abcdef ghjijklmn & 123.456778 \\
One & abcdef ghjijklmn & 123.456778 \\
One & abcdef ghjijklmn & 123.456778 \\
One & abcdef ghjijklmn & 123.456778 \\
One & abcdef ghjijklmn & 123.456778 \\
One & abcdef ghjijklmn & 123.456778 \\
One & abcdef ghjijklmn & 123.456778 \\
One & abcdef ghjijklmn & 123.456778 \\
One & abcdef ghjijklmn & 123.456778 \\
One & abcdef ghjijklmn & 123.456778 \\
\end{longtable}
\end{center}