\chapter{Einleitung}

In dieser Arbeit gedenken die Autoren die 4-Tage-Woche zu untersuchen. 
Die 4-Tage-Woche ist ein Arbeitszeitmodell, bei dem Arbeitnehmer anstatt fünf Tagen nur vier Tage pro Woche arbeiten. 
Dabei gibt es im allgemeinen drei verbreitete Modelle. Im ersten Modell wird lediglich die be-stehende Wochenarbeitszeit auf vier anstelle fünf Tage aufgeteilt. 
Im zweiten Modell wird die Wochenarbeitszeit reduziert, jedoch auch der Lohn. Im dritten Modell wird die Wochenarbeitszeit reduziert, der Lohn bleibt allerdings gleich. \parencite[vgl.][]{habdank_deutscher_2024}

\section{Motivation}

Die traditionelle Arbeitswoche, geprägt von festen Arbeitszeiten und starren Strukturen, steht nicht erst seit kurzem im Fokus von 
Diskussionen über Flexibilisierung und Modernisierung. Die Notwendigkeit, Arbeitszeitmodelle anzupassen, wird immer deutlicher, da sich 
die Anforderungen und Erwartungen von Arbeitnehmern im Laufe der Zeit verändern. Unter anderem die Corona Pandemie hat diesen Wandel in 
besonderem Maße befeuert. Das Ergebnis macht sich beispielweise in der vermehrten Selbstorganisation unter Arbeitnehmern und durch den 
durchschlagenden Erfolg des Home-Office bemerkbar. \parencite[vgl.][S. 73]{haide_arbeitswelt_2022}
In einer Zeit, in der Arbeitgeber vermehrt einen Rückgang der während der Corona-Pandemie gewährten Flexibilität anstreben (\cite{elias_googles_2023}; \cite{lee_apple_2022}; \cite{vanian_meta_2023}), 
gewinnt das Thema „New Work“ und besonders Konzepte wie die 4-Tage-Woche wieder an Bedeutung.
Insbesondere die junge Generation - Generation Z - drängt verstärkt auf eine verbesserte Work-Life-Balance und sucht nach Arbeitsmodellen, 
die es ermöglichen, berufliche Verpflichtungen mit persönlichen Interessen und familiären Verpflich-tungen in Einklang zu bringen. \parencite[vgl.][S. 10]{onlyfy_wechselwilligkeitsstudie_2023}

\section{Zielsetzung}

Um ein umfassendes Verständnis für die Einstellung zur 4-Tage-Woche und ihre potenziellen Auswirkungen zu gewinnen, 
hat sich das erste Semester des Masterkurs IT-Management der Hochschule Mainz zum Ziel gesetzt, dieses Thema genauer zu untersuchen. 
Durch eine sorgfältige Analyse sollen verschiedene Thesen bezüglich der Einstellung zur verkürzten Arbeitswoche allgemein sowie 
ihrer Auswirkungen auf die Arbeitgeber sowie Arbeitnehmer überprüft werden. Dabei werden verschiedene Aspekte der Arbeitswelt 
betrachtet, von der Produktivität und Effizienz bis hin zur Mitarbeiterzufriedenheit und -bindung. Ziel ist es, die 
vielschichtigen Herausforderungen und Chancen, die mit der Einführung einer 4-Tage-Woche einhergehen, genauer zu identifizieren 
und offenlegen zu können.

\section{Aufbau der Arbeit}

Im folgenden Kapitel wird zunächst die Durchführung der Umfrage und die Datenaufbereitung beschrieben. Darauf aufbauend widmen sich die Autoren der deskriptiven Analyse der erhobenen Daten.
Ziel dessen ist es, in den folgenden Kapiteln die ausgewählten Hypothesen zu überprüfen.

In einer abschliesenden Diskussion werden die Ergebnisse der Hypothesenüberprüfung zusammengefasst und interpretiert und darauf aufbauend ein Fazit gezogen sowie ein Ausblick gegeben.