% Erklärender Text zu dieser Datei --------------------------------------------------------
% Die Datei 00_Inhalt_Expose.tex dient dem Schreiben des Expose.
% Die Vorlage enthält bereits typische Abschnitte eines Exposes, sodass diese nur noch ausgefüllt werden müssen.
%
% Weitere Abschnitte sind möglich und sollten mit dem Betreuer abgestimmt werden.
% Beispiele für weitere Abschnitte sind: Relevanz und Aktualität des Themas, Theoretischer Hintergrund oder Zeitplanung
% -----------------------------------------------------------------------------------------

\section*{Beschreibung des Themas}
In diesem Abschnitt wird das Thema in mindestens 2-3 Sätzen beschrieben. 
Ziel ist es ein Verständnis für das Thema zu geben und dieses grob zu verorten.

\section*{Forschungsfrage}
In diesem Abschnitt wird die Forschungsfrage der Arbeit formuliert. 
Diese sollte von praktischer oder wissenschaftlicher Relevanz sein und einen allgemeinen Charakter haben.
Im Zuge einer Forschungsfrage können auch weitere Fragestellungen betrachtet werden, die gelistet werden sollten. 
\begin{itemize}
    \item Untergeordnete Fragestellung 1
    \item Untergeordnete Fragestellung 2
    \item Untergeordnete Fragestellung 3
\end{itemize}

\section*{Methodik und Vorgehen}
Beschreibung der Vorgehensweise, die zur Beantwortung der Forschungsfrage genutzt wird. 
Beispielsweise Literaturuntersuchung, Implementierung von Algorithmen oder die Auswertung von Fallbeispielen.

\pagebreak
\section*{Vorl"aufige Gliederung}
% Die nachfolgende Gliederung ist nur ein Beispiel, damit die Befehle zur Erstellung der Gliederung gezeigt werden können.
% Die folgenden zwei Zeilen dienen der richtigen Nummerierung der Aufzählungsebenen
\renewcommand{\labelenumii}{\arabic{enumi}.\arabic{enumii}}
\renewcommand{\labelenumiii}{\arabic{enumi}.\arabic{enumii}.\arabic{enumiii}}
\begin{enumerate}
    \item Einleitung
        \begin{enumerate}
        \item Motivation
        \item Ziele und Forschungsfrage
        \item Aufbau der Arbeit
        \end{enumerate}
    \item Grundlagen
        \begin{enumerate}
        \item Reinforcement Learning
            \begin{enumerate}
            \item Begriffserkl"arung und Verortung
            \item Markov Decision Process
            \item Dynamic Programming
            \item Temporal Difference Learning
            \item Exploitation vs Exploration
            \end{enumerate}
        \item Q-Learning
        \item SARSA      
        \item MiniMax-Algorithmus
        \item Tic-Tac-Toe
            \begin{enumerate}
            \item Spielerkl"arung
            \item Anwendbare Reinforcement Learning Algorithmen
            \end{enumerate}
        \end{enumerate}
    \item Methodik und Funktionsweise der Algorithmen
        \begin{enumerate}
        \item Spielfeld
        \item Agent Q-Learning
        \item Agent SARSA
        \item Trainingsaufbau und Evaluationsmetriken
        \end{enumerate}
    \item Implementierung
    \item Diskussion und Auswertung der Ergebnisse
        \begin{enumerate}
        \item Auswertung des Q-Learning Agenten
        \item Auswertung des SARSA Agenten
        \end{enumerate}
    \item Konklusion
        \begin{enumerate}
        \item Beantwortung der Forschungsfragen
        \item Kritische Betrachtung der Inhalte
        \item Anmerkungen f"ur k"unftige Arbeiten
        \end{enumerate}   
\end{enumerate}

