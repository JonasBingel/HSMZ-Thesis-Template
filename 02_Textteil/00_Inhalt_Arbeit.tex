% Erklärender Text zu dieser Datei --------------------------------------------------------
% Die Datei 00_Inhalt_Arbeit dient als zentrale Datei, in der die Struktur der Arbeit definiert und einzelne Abschnitte eingeladen werden.
%
% Es wird empfohlen für jedes Kapitel bzw. Abschnitt der Arbeit eine separate Datei anzulegen.
% Die einzelnen Dateien sind dann mittels der Anweisung \include in dieser Datei einzuladen.
% Ein Beispiel für dieses Vorgehen bietet die Bachelorarbeit, die im Repository der Vorlage verlinkt ist.
%
% Derzeit enthält die Datei Beispiele für die Features der LateX-Vorlage.
% -----------------------------------------------------------------------------------------
\chapter{Beispiele}

Dieses Kapitel enthält einige Beispiele , wie in dieser \LaTeX-Vorlage Artefakte dargestellt und referenziert werden können.
Zunächst soll die Verwendung von Abkürzungen gezeigt werden am Beispiel von \ac{API} und \ac{ML}.
Insbesondere \ac{ML} ist dabei interessant, da dort noch eine deutsche Übersetzung angegeben ist. Abkürzungen werden im Vortext in \texttt{Abkuerzungen.tex} definiert und automatisch in das Abkürzungsverzeichnis aufgenommen, wenn diese in der Arbeit verwendet werden.

Die Zitation in dieser Arbeit erfolgt mit dem Befehl \texttt{cite} \cite[S. 0]{ertelIntroductionArtificialIntelligence2017}.
Der Zitationsstil kann in \texttt{Packages.tex} geändert werden beim Import des Packages \texttt{biblatex}.
Als Standard ist derzeit IEE eingetragen. 
Notwendige Änderungen zur Verwendung von APA sind als Kommentar dargestellt.
Die Änderung des Zitationsstils beeinflusst auch die Darstellung im Literaturverzeichnis.
In das Literaturverzeichnis werden nur die Quellen aufgenommen, die im Text zitiert werden.
Die Quellen werden in \texttt{Bachelorarbeit.tex} eingeladen.
Außerdem wird dort durch den Befehl \texttt{notcite} das Verhalten geändert, sodass alle Quellen gelistet werden.
Dies ist gut zur Pflege des Literaturverzeichnisses während der Bearbeitung, aber sollte bei der Abgabe entfernt werden.

Fußnoten können ebenfalls gesetzt werden mit dem Befehl \texttt{footnote}.\footnote{Fußnoten funktionieren}
Soll eine Fußnote in der Caption eines Artefakts verwendet werden, um deren Quelle anzugeben, ist ein Workaround notwendig.
Dieser Workaround wird am Beispiel von \cref{fig:example} vorgestellt. Zur Übersichtlichkeit wurde für Artefakte ein Verzeichnis 04\_Artefakte angelegt.

% In der Caption der Figure wird mit \protect\footnotemark eine Fußnote platziert. Der Text wird außerhalb der figure-Umgebung hinzugefügt mittels \footnotetext
\begin{figure}[h]
\centering
\rule{.5\textwidth}{1cm}
\caption[{Beispiel Abbildung}]{Beispiel Abbildung mit Fußnote\protect\footnotemark}
\label{fig:example}
\end{figure}
\footnotetext{entnommen aus \cite[S. 0]{ertelIntroductionArtificialIntelligence2017}}

Neben Abbildungen können auch Tabellen erstellt werden, wíe \cref{tab:shortsample}. 
Zur leichteren Erstellung von Tabellen gibt es im Internet entsprechende Tools. 
Diese müssen dann nur noch geringfügig angepasst werden. 
Im \cref{app:longtable} ist ein Beispiel für eine Tabelle, die größer als eine Seite ist und mittels \texttt{longtable} erstellt wurde.

\begin{table}[ht]
\centering
\caption{Kurze Beispieltabelle}
\label{tab:shortsample}
\begin{tabular}[t]{lcc}
\toprule
&Treatment A&Treatment B\\
\midrule
John Smith&1&2\\
Jane Doe&--&3\\
Mary Johnson&4&5\\
\bottomrule
\end{tabular}
\end{table}%

Außerdem können Gleichungen notiert werden, wie in \cref{eq:culomb} das Coloumb-Gesetz.
Diese Gleichung wurde gewählt, um das Symbolverzeichnis zu füllen. 
Die eigentliche Definition der Nomenklatur erfolgt in \texttt{Nomenklatur.tex} im \texttt{Verzeichnis 00\_Allgemein}.

\begin{equation}
    \label{eq:culomb}
    \equationentry{Coulomb-Gesetz}
    F = \frac{1}{4\pi\epsilon_0}\cdot\frac{Q_1Q_2}{r^2}
\end{equation}

Das nächste Artefakt ist ein Algorithmus.
Die Darstellung von Algorihtmen erfolgt in dieser Arbeit mittels dem Package Algorithm2e.
Algorithmen können ebenfalls referenziert werden, jedoch muss der Befehl \texttt{ref} genutzt werden.
Ein Beispiel ist der Algorithmus \ref{alg:howto}.

\begin{algorithm}[H]
\caption{How to write algorithms}
\label{alg:howto}

\SetAlgoLined
\KwData{this text}
\KwResult{how to write algorithm with \LaTeX2e }
initialization\;
\While{not at end of this document}{
read current\;
\eIf{understand}{
go to next section\;
current section becomes this one\;
}{
go back to the beginning of current section\;
}
}

\end{algorithm}

Zum Einbinden von Source code wird das Paket minted genutzt. 
Für die Programmiersprache Java ist bereits ein Stil zur Darstellung in \texttt{Packages.tex} definiert.
Weitere Programmiersprachen können genau so definiert werden. Entsprechende Hinweise gibt es in der 
\href{https://www.ctan.org/pkg/minted}{Dokumentation von minted}.
In \cref{listing:1} wird ein simples HelloWorld-Programm gezeigt, während \cref{app:longlisting} ein längeres Beispiel zeigt, dass über mehrere Seiten geht.

\begin{listing}[h]
\caption{Hello World Example in Java}
\label{listing:1}
\inputminted{java}{04_Artefakte/03_Listings/HelloWorld.java}
\end{listing}

\blinddocument


