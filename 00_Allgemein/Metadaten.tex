% Erklärender Text zu dieser Datei --------------------------------------------------------
% Die Datei Metadaten.tex dient der Definition von Metadaten zur vorliegenden Arbeit.
% Dies umfasst Informationen zu
%  - der Arbeit selbst(Titel, Art)
%  - dem Betreuer
%  - dem Hochschule
%  - ggf. kooperierenden Unternehmen
% Die Metadaten werden genutzt in Titelseite.tex, Sperrvermerk.tex und Erklaerung.tex
% Ferner werden die Metadaten des exportierten PDFs entsprechend gesetzt - eine Konfiguration ist in Packages.tex möglich.
% -----------------------------------------------------------------------------------------

% Meta-Daten zu dieser Arbeit ------------------------------------------------------------
\newcommand{\art}{Forschungsbericht\xspace}
\newcommand{\titel}{Statistische Auswertung einer Umfrage zur "4-Tage-Woche"\xspace}
\newcommand{\hochschule}{Hochschule Mainz}
\newcommand{\hochschulezusatz}{University of Applied Sciences}
\newcommand{\logo}{logo_hsmz.png}
\newcommand{\fachbereich}{Wirtschaft}
\newcommand{\studiengang}{IT-Management M.Sc. berufsintegriert}

\newcommand{\autor}{Philipp Gremeyer Patrick Pfurtscheller und Subhan Ullah}
\newcommand{\strasseAutor}{Straße}
\newcommand{\stadtAutor}{PLZ Ort}
\newcommand{\matrikelnr}{123456}

\newcommand{\unternehmen}{Unternehmen}
\newcommand{\datumAblaufSperrvermerk}{01.01.1970}

\newcommand{\betreuer}{Prof. Dr. Anett Mehler-Bicher}
\newcommand{\datumAbgabe}{31.07.2024}
\newcommand{\ort}{Mainz}

