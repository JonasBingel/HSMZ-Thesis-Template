% Erklärender Text zu dieser Datei --------------------------------------------------------
% Die Datei Befehle.tex dient der Definition von Custom-Anweisungen, sodass diese in den restlichen Dateien genutzt werden können. 
% Custom-Anweisungen eigenen sich inbesondere als Variablen für Fachbegriffe, sodass sichergestellt ist, dass diese in der gesamten Arbeit einheitlich geschrieben sind und zentral geändert werden können.
% Zudem enthält die Datei bereits Anweisungen für häufig genutzte Abkürzungen.
% -----------------------------------------------------------------------------------------

% Befehle zu Abkuerzungen --------------------------------------------------------------
% Die gelisteten Befehle für Abkürzungen stammen aus github.com/StefanMacke/latex-vorlage-fiae/blob/master/Allgemein/Befehle.tex

% Die Anweisung \, erzeugt einen kurzen Abstand und wird bei Abkuerzungen oder zwischen Zahlen und Masseinheiten verwendet
\newcommand{\bs}{$\backslash$\xspace}
\newcommand{\ua}{\mbox{u.\,a.}\xspace}
\newcommand{\oa}{\mbox{o.\,a.}\xspace}
\newcommand{\bspw}{bspw.\xspace}
\newcommand{\bzw}{bzw.\xspace}
\newcommand{\ca}{ca.\xspace}
\newcommand{\dahe}{\mbox{d.\,h.}\xspace}
\newcommand{\etc}{etc.\xspace}
\newcommand{\eur}[1]{\mbox{#1\,\texteuro}\xspace}
\newcommand{\evtl}{evtl.\xspace}
\newcommand{\ggfs}{ggfs.\xspace}
\newcommand{\Ggfs}{Ggfs.\xspace}
\newcommand{\gqq}[1]{\glqq{}#1\grqq{}} % Anführungszeichen oben und unten
\newcommand{\idR}{i.d.R.\xspace}
\newcommand{\inkl}{inkl.\xspace}
\newcommand{\insb}{insb.\xspace}
\newcommand{\usw}{usw.\xspace}
\newcommand{\Vgl}{Vgl.\xspace}
\newcommand{\sogn}{sogn.\xspace}
\newcommand{\zB}{\mbox{z.\,B.}\xspace}
\newcommand{\engl}{engl.\xspace}
\newcommand{\dt}{dt.\xspace}

% Befehle zu Fachbegriffen --------------------------------------------------------------
% Es ist empfehlenswert alle Fachbegriffe als Anweisung zu deklarieren, damit diese einheitlich sind und zentral geändert werden können

% In der folgenden Zeile ein Beispiel für den Begriff "Q-Learning", sodass im Text die Anweisung \qlearning genutzt werden kann, welche zu "Q-Learning" übersetzt wird 
%\newcommand{\qlearning}{Q-Learning\xspace}


% Custom Anweisungen ------------------------------------------------------------------
% Anweisung um mehrere PDF-Seiten einzufügen
\newcommand\includepdfWithChapter[4]{%
\includepdf[pages={#1}, pagecommand={\chapter{#3}}]{#4}
\includepdf[pages={#2}]{#4}
}

